\include{preamble.tex}

\begin{document}

\noindent
   \begin{center}
   \framebox{
      \vbox{
    \hbox to 5.78in { {\bf CS 530: Advanced Algorithms} \hfill  }
       \vspace{4mm}
       \hbox to 5.78in { {\Large \hfill Homework 1: Review and Intro to Approximation Algorithms \hfill} }
             \vspace{2mm}
       \hbox to 5.78in { {\it Fall 2025 \hfill } }
      }
   }
   \end{center}

\pagenumbering{gobble}

\section{Problem 1: TSP (10 points)} 

\begin{enumerate}[(a)]
\item Show that metric TSP is equivalent to the following problem: given a weighted graph $G$, find the cheapest walk that visits every vertex at least once and returns to the starting point.
\item In the metric path TSP problem, we are given special vertices $s$ and $t$ and need to compute the cheapest Hamiltonian path starting at $s$ and ending at $t$. Give a 2-approximation for this problem.
\end{enumerate}

\section{Problem 2: Knapsack (15 points)}

\begin{enumerate}[(a)]
\item Modify the algorithm we went over in class so that it runs in time $O(n \cdot \min(W,V))$, where $W$ is the capacity of the knapsack and $V$ is the value of the optimal solution. \textbf{Hint:} What solutions do you really need to keep around in each subproblem?
\item Modify the algorithm so that it also returns an optimal solution, not just its value.
\item Show how to modify the FPTAS to improve the running time to $O(n^2/\epsilon)$. \textbf{Hint:} Start by running the greedy 2 approximation for knapsack.
\end{enumerate}

\section{Problem 3: Implementation of Knapsack (10 points)}

Implement the dynamic program for knapsack you obtained in 2(a) and run it on the instance posted \href{https://nathan-klein.github.io/algs-materials/knapsack.py}{here} for bounds $W=10000000$ and $W=20000000$. Turn in your code and write the answers you got here.

\end{document}