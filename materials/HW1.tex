\include{preamble.tex}

\begin{document}

\noindent
   \begin{center}
   \framebox{
      \vbox{
    \hbox to 5.78in { {\bf CS 599: Rounding Techniques in Approximation Algorithms} \hfill  }
       \vspace{4mm}
       \hbox to 5.78in { {\Large \hfill Homework 1: Intro and Independent Randomized Rounding \hfill} }
             \vspace{2mm}
       \hbox to 5.78in { {\it Fall 2024 \hfill } }
      }
   }
   \end{center}

\pagenumbering{gobble}

\section{Problem 1: Vertex Cover} 

\begin{enumerate}
\item Write the natural integer programming relaxation for Vertex Cover and the corresponding relaxed linear program.
\item Prove that if $x$ is an extreme point solution to this LP, then $x_v \in \{0,\frac{1}{2},1\}$ for all $v \in V$.
\item Give a $\frac{3}{2}$ approximation for Vertex Cover on planar graphs. Show a matching lower bound on the integrality gap.
\item Given an example that shows the greedy algorithm for vertex cover (iteratively picking the max degree vertex and deleting it and its edges from the graph) has approximation ratio $\Omega(\log n)$. 
\item \textit{(Bonus question)} Give a lower bound showing that the Vertex Cover LP with triangle constraints (see Lecture 1) still has an integrality gap of 2 (up to lower order terms).
%\item Give the exact integrality gap for graphs of bounded degree $d$ (for every $d$). 
\end{enumerate}

%\section{Problem 2: MAX $k$-CUT}
%
%In MAX $k$-CUT, we are given a graph $G=(V,E)$ and the goal is to partition it into $k$ pieces $V_1,\dots,V_k$ so as to maximize the number of edges with endpoints in different parts. 
%\begin{enumerate}
%\item Give a randomized $\frac{k-1}{k}$ approximation for this problem. 
%\item Derandomize this algorithm to obtain a deterministic $\frac{k-1}{k}$ approximation.
%\end{enumerate}

\section{Problem 2: Chernoff Bounds}

\begin{enumerate}
%\item Consider the coupon collector problem: there are $n$ items and at each time step $1,2,\dots$ we collect a uniformly random item. Let $X$ be the timestep all items are collected. Prove that  $\E{X} = (1+o(1))n \log n$. Furthermore, show that $\P{|X - \E{X}| \ge \epsilon n} \le e^{-\Omega(\epsilon^2)}$ for $\epsilon \ge 2$.
\item Prove that the congestion for the multi-commodity flow problem we discussed in Lecture 2 can be improved to $O(\log n/\log \log n)$. 
\item Prove that the maximum degree of a vertex in a uniformly random spanning tree on the complete graph is $O(\log n/\log \log n)$ with high probability. You may use the \href{https://en.wikipedia.org/wiki/Pr\%C3\%BCfer_sequence}{Prüfer code}. 
\end{enumerate}

\section{Problem 3: Set Cover}

Set cover is a generalization of vertex cover. Here we are given elements $E = \{e_1,\dots,e_n\}$ and some sets $S_1,\dots,S_m \subseteq E$ with non-negative weights $w_1,\dots,w_m \in \R_{\ge 0}$. Our goal is to select a collection of sets $\cS$ with minimum cost that covers all the elements, i.e. we should have $\bigcup_{S \in \cS} S_i = E$.
\begin{enumerate}
\item Write the natural integer programming relaxation for set cover and the corresponding relaxed linear program. 
\item Prove that the probability a given element is covered by including each set $S_i$ in $\cS$ independently with probability $x_{i}$ is at least $1-1/e$. 
\item Give a $O(\log n)$ approximation for set cover that works with probability at least $1-1/n$. 
\item Prove that the integrality gap of the LP for set cover is at least $(1-\epsilon)\ln n$ for every $\epsilon > 0$. \textbf{Hint:} Use a random instance. 

\item \textit{(Bonus question)} Give a deterministic construction with gap $\Omega(\log n)$. 
\end{enumerate}


%\section{Problem 3: $s$-$t$ Minimum Cut}
%
%In the $s$-$t$ minimum cut problem, we are given a graph $G=(V,E)$ and two distinct vertices $s,t \in V$. The goal is to partition the graph into two pieces, $S$ and $V \smallsetminus S$ such that $s \in S$ and $t \not\in S$, so that $|\delta(S)|$ is minimized, where $\delta(S)$ is the set of edges with exactly one endpoint in $S$. 
%
%\begin{enumerate}
%\item Write the natural linear programming relaxation for $s$-$t$ minimum cut. 
%\item Give a randomized 1-approximation for $s$-$t$ minimum cut. 
%\item Show that any extreme point to the LP is actually an integer point. 
%\end{enumerate}

%\section{Problem 3: MAX SAT}
%
%In the MAX SAT problem, we are given $n$ variables $x_1,\dots,x_n$ which we can set to true or false. Furthermore, we are given $m$ clauses $C_1,\dots,C_m$ which consist of unnegated and negated variables, so that a clause $C_j$ consists of two sets of indices, $P_j$ and $N_j$, and is of the form $$\bigvee_{i \in P_j} x_i \lor \bigvee_{i \in N_j} \overline{x_i}.$$
%We may assume $P_j \cap N_j = \emptyset$ and $|P_j \cup N_j| \ge 1$ for all clauses.
%
%\begin{enumerate}
%\item Write an integer programming relaxation of MAX SAT and the corresponding relaxed linear program.
%\item Give a randomized 	$(1-\frac{1}{e})$ approximation for MAX SAT.
%\end{enumerate}

\end{document}