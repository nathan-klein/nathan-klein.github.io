\include{preamble.tex}

\renewcommand{\P}[1]{{\mathbb{P}}\left[#1\right]}
\renewcommand{\PP}[2]{{\mathbb{P}}_{#1}\left[#2\right]}
\renewcommand{\E}[1]{{\mathbb{E}}\left[#1\right]}
\renewcommand{\EE}[2]{{\mathbb{E}}_{#1}\left[#2\right]}
\renewcommand{\R}{\ensuremath{\mathbb R}}
\renewcommand{\Z}{\ensuremath{\mathbb Z}}
\def\cI{{\cal I}}

\begin{document}

\noindent
   \begin{center}
   \framebox{
      \vbox{
    \hbox to 5.78in { {\bf CS 599: Rounding Techniques in Approximation Algorithms} \hfill  }
       \vspace{4mm}
       \hbox to 5.78in { {\Large \hfill Homework 2: Dependent Randomized Rounding \hfill} }
             \vspace{2mm}
       \hbox to 5.78in { {\it Fall 2024 \hfill } }
      }
   }
   \end{center}


\section{Problem 1: P=NP...?} 

\begin{enumerate}
	\item In the traveling salesperson problem, we are given a complete graph $G=(V,E)$ with edge costs $c: E \to \R_{\ge 0}$ that form a metric, i.e. $c_{\{u,w\}} \le c_{\{u,v\}}+c_{\{v,w\}}$ for all $u,v,w$. Our goal is to find the minimum cost Hamiltonian cycle. Now where $n=|V|$ and $E(S)$ for $S \subseteq V$ is the set of edges with both endpoints in $S$, let $$P_{\mathrm{sub}} = \begin{cases} \sum_{e \in E} x_e = n \\ \sum_{e \in E(S)} x_e \le |S|-1 & \forall S \subsetneq V \\ x_e \ge 0 & \forall e \in E \end{cases}$$
	Prove that $P_{\mathrm{sub}} \cap \{0,1\}^E$ is the set of all feasible solutions to the traveling salesperson problem, i.e. all Hamiltonian cycles in $G$. 
	\item Notice that $P_{\mathrm{sub}}$ is exactly the spanning tree polytope $P_{\mathrm{st}}$ except we have changed the $n-1$ to an $n$. So, it seems like we should be able to apply the proof from Lecture 4 to show that it has integral vertices. Either use this to prove P=NP or find a flaw in the argument from Lecture 4 when applied to $P_{\mathrm{sub}}$ instead of $P_{\mathrm{st}}$ (i.e. when we change the $n-1$ to an $n$).
	\item In Lecture 3, we mentioned that it is NP-Hard to obtain a $1.001$ approximation for weighted $k$-ECSS for any $k$. Either explain where the proof fails and give an explicit counterexample showing that the $1+O(\sqrt{\frac{\log n}{k}})$ approximation does not work for weighted $k$-ECSS, or adapt the algorithm to prove that P=NP.  
\end{enumerate}

\section{Problem 2: Scaling into Integral}

Suppose $P$ is a polytope in $[0,1]_{\ge 0}^n$ and $\tilde{P}$ is the convex hull of $P \cap \Z^n$. Now, suppose that there is an $\alpha \ge 1$ such that given any point $x \in P$ there exists a randomized algorithm $A$ that produces a random point $\tilde{x} \in \tilde{P}$ such that $\E{\tilde{x}_i} \le \alpha x_i$ for every input, where the expectation is taken over the possible outputs $\tilde{x}$ of $A$ given $x$. 

Given a polytope $P$, $P^\uparrow$ is called the \textit{dominant} of $P$ and consists of all points $x$ such that there exists $x' \in P$ for which $x'-x \in \R^n_{\ge 0}$.
\begin{enumerate}
	\item Prove that $\alpha \cdot P \subseteq \tilde{P}^\uparrow$ (where $\alpha \cdot P$ consists of all points in $x$ scaled entry-wise by $\alpha$).
	\item Prove that if $P$ is defined by the constraints $0 \le x_i \le 1$ and a collection of \textit{covering constraints}, i.e. constraints of the form $a^Tx \ge b$ for $a \in \R_{\ge 0}^n, b \in \R_{\ge 0}$, then the integrality gap of the LP $\min c^Tx$ subject to $x \in P$ is at most $\alpha$ for any $c \in \R^n_{\ge 0}$.
\end{enumerate}

\section{Problem 3: Randomized Pipage Rounding for Matroids}

A \textit{matroid} $M=(E,\cI)$ is defined by a collection of elements $E$ and a collection of \textit{independent sets} $\cI \subseteq 2^E$ with $\emptyset \in \cI$ and the properties:
\begin{enumerate}[(i)]
\item \textbf{Downward Closed}: If $I \in \cI$, then $J \in \cI$ for every $J \subseteq I$.
\item \textbf{Augmentation Property}: If $I,J \in \cI$ and $|I| < |J|$ then there exists some $e \in J$ such that $I \cup \{e\} \in \cI$. 
\end{enumerate}
A \textit{basis} of a matroid is any maximal independent set. The \textit{rank} of a collection of elements $F \subseteq E$ is the maximum possible size of $I \cap F$ for any $I \in \cI$. Let $r: 2^E \to \Z_{\ge 0}$ be the rank function.

In this problem, we will consider the following polytope $P_M$ for a matroid $M$:
$$P_M = \begin{cases} x(E) = r(E) \\ x(S) \le r(S) & \forall S \subseteq E \\ x_e \ge 0 & \forall e \in E \end{cases}$$
\begin{enumerate}
\item Argue that the set of forests of a graph forms a matroid $M$, and that in this case $P_M = P_{\mathrm{st}}$.
\item Adapt the proof in class for the spanning tree polytope to show that for every matroid $M$, $P_M$ as defined above is the convex hull of its (integral) bases, i.e. it is the base polytope of $M$. You may use that the rank function $r$ is submodular, i.e. $r(S)+r(T) \ge r(S \cup T) + r(S \cap T)$ for $S,T \subseteq E$. Then briefly argue that randomized pipage rounding works for any matroid with the same guarantees as for spanning trees.
\item Use the \href{https://en.wikipedia.org/wiki/Method_of_conditional_probabilities}{method of conditional expectation} to derandomize randomized pipage rounding so that given $x \in P_M$ and any cost function $c:E \to \R^E$, we can find a point of cost at most $c(x)$ in polynomial time. 
\item When using given an independent distribution $\mu: 2^E \to \R^E_{\ge 0}$ with marginals $x$, we have used that $\PP{S \sim \mu}{|S \cap F|=0} \le e^{-x(F)}$ for any $F \subseteq E$. Show that this also holds for distributions with 0-negative correlation and thus for randomized pipage rounding. 

Show that the same bound does \textit{not} hold for all distributions $\mu$ with only negative correlation, i.e. $\E{\prod_{i \in S} X_i} \le \prod_{i \in S} \E{X_i}$ for all $S \subseteq E$, by exhibiting a negatively correlated distribution for which this does not hold. 
\end{enumerate}

\section{Problem 4: Lottery}

Use the above to show that we can design a multi-item lottery as follows. Suppose we have a collection of $n$ goods $g_1,\dots,g_n$, a collection of $m$ people $w_1,\dots,w_m$, and a specified $0 < \epsilon < 1$. Now, we will allow people to buy up to one lottery ticket. Before they purchase a ticket, they will specify the subset of the goods that they would be happy winning, and if we sell them a ticket we must promise them that their chance of getting one of their chosen goods is at least $\epsilon$. 

Using that $P_M$ has a polynomial time separation oracle for every $M$, implement a polynomial time lottery system that (i) can determine when a ticket can be faithfully sold (this will depend on the subset of goods desired), (ii) will output a random assignment that respects the guarantee, and (iii) is \textit{fair to all groups} in the sense that for any group of $k$ people, the probability at least one of them wins something is at least $1-e^{-k\epsilon}$.

\section{Bonus Problems} 

\begin{enumerate}
	\item Prove the parsimonious property from Lecture 6 using splitting off. 
	\item Show that the randomized rounding algorithm for the multi-commodity flow problem can be derandomized using the method of conditional expectation.
	\item Give an example showing that the Chernoff bound upper tail does not hold for distributions with only \textit{pairwise} negative correlation, i.e. $\P{e,f \in S} \le \P{e \in S}\P{f \in S}$ (and not full negative correlation a.k.a the NCP as we proved for pipage rounding). 
	\item (***) Prove that swap rounding gives a $1+O(1/\sqrt{k})$ for $k$-ECSM. (Note: This is quite difficult. I think I vaguely see how to prove this but I'm not 100\% sure. This could potentially even be false, although I don't think so. This could be an interesting final project.)
\end{enumerate}

\end{document}