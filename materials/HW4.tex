\include{preamble.tex}

\renewcommand{\P}[1]{{\mathbb{P}}\left[#1\right]}
\renewcommand{\PP}[2]{{\mathbb{P}}_{#1}\left[#2\right]}
\renewcommand{\E}[1]{{\mathbb{E}}\left[#1\right]}
\renewcommand{\EE}[2]{{\mathbb{E}}_{#1}\left[#2\right]}
\renewcommand{\R}{\ensuremath{\mathbb R}}
\renewcommand{\Z}{\ensuremath{\mathbb Z}}
\def\cI{{\cal I}}

\begin{document}

\noindent
   \begin{center}
   \framebox{
      \vbox{
    \hbox to 5.78in { {\bf CS 599: Rounding Techniques in Approximation Algorithms} \hfill  }
       \vspace{4mm}
       \hbox to 5.78in { {\Large \hfill Homework 4: Iterative Randomized Rounding and SDPs \hfill} }
             \vspace{2mm}
       \hbox to 5.78in { {\it Fall 2024 \hfill } }
      }
   }
   \end{center}

\pagestyle{empty}

\vspace*{3mm}
\noindent {\large \textcolor{purple}{Do at least two of the following three problems.}}

\section{Problem 1: Beck-Fiala}

Use the iterative randomized rounding framework to prove a bound of $O(\sqrt{t \log n})$ for Beck-Fiala, where $n$ is the number of edges and the hypergraph has maximum degree $t$. 

\section{Problem 2: Matroid Intersection with Concentration}

Let $M_1=(\cI_1,E)$ and $M_2=(\cI_2,E)$ be two matroids over the same ground set $E$. Let $x$ be a point in the matroid intersection polytope $P_{M_1 \cap M_2}$. Give a polynomial algorithm that given $x$ produces a set $I \subseteq E$ such that:
\begin{enumerate}
	\item $c(I) \le 2c(x)$,
	\item $I$ contains a basis of $M_1$ and $M_2$,
	\item For any set of elements $F \subseteq E$, $\P{|F \cap I| \ge 2(1+\epsilon) \cdot  x(F)} \le e^{-\Omega(\epsilon^2 \cdot x(F))}$.
\end{enumerate}
\textbf{Hint 1:} Use the iterative randomized relaxation framework with a slight twist.\\

\noindent\textbf{Hint 2:} The set of constraints in $P^\uparrow_M$ can be uncrossed to form a chain for any matroid similar to the constraints for $P_M$. You may use this as a given if you'd like.


\section{Problem 3: SDPs}

Read Section 13.2 of the Williamson-Shmoys book and show that the algorithm there can be used to color a 3-colorable graph with $\tilde{O}(n^{1/4})$ colors (where $\tilde{O}$ hides polylogarithmic factors), i.e. do Exercise 13.1 in the Williamson-Shmoys book. 

\end{document}